\documentclass[12pt]{article}

\title{Reading Response Prompts (Ch. 1-5)}
    \author{Eddie Englund}

\begin{document}
    
    \maketitle

    \section{Pre-Reading}

    \begin{enumerate}
        \item I don't believe that I will enjoy this story as I prefer reading things with greater depth, character and especially adventure. However, while I do see the meaning of making fables to teach lessons I find most of the boring or they're trying to portray a picture that I already know.
        \item No, I do not. If you make the story about people most folks who read it will not be able to see the grand picture and start to morally and emotionally align with one part or another.
        \item Pig's in almost the entire world, are viewed as nasty, dirty, tainted, fat and lazy. Because of this, it's usually what we depict our enemies as. For example, saying ``You're such a pig'' usually means that either you're not keeping a location like your room clean or you're not grooming yourself enough. This is probably what the writer wants to sub-consciously
    \end{enumerate}

    \section{Chapter 1}

    \begin{enumerate}
        \item The old major's evidence for believing that humans are their enemy is quite lackluster and not very thought through. He begins by calling out all the ``evils of man'' like, how they took the horses foul and the Hens eggs. The issue with this view is that he does not know the reason. He simply stated the \textit{evil's} of the owner Mr. Johnson but did not incorporate his reasoning. It also assumes that just because Mr.Johnson was ``evil'' all humans must be so as well.
        \item A world controlled by workers will not work. This has been proven many times. The issue at hand is that there as simply too many differences between people and they must be lead by someone or a group of people. The reason for this is that with too many different variables in different people will lead to different ways of execution and by controlling this with a leader-worker basis it works a lot better. The major variable with people is knowledge and personality. People with greater knowladge are usually better leaders however, they just like the workers must be kept in check so that they do not exceed their boundaries.
        \item As stated above humans are far from perfect. If we were perfect we would not have a mental illness and other forms of suffering. The answer is short and simple. No. There is no group of people or animals that can have a utopia. There are simply too many variables, you simply just have to make the best of what you got.
        \item In the first chapter of the book, they have a vote on if the rats should be comrades too. This inherently disproves what the old major said before about how ``all the animals are equal''. If they were equal there wouldn't be a need for a vote.
        \item The first thing I would lay out in the debate would be that I, just as he works hard every day. I'd also say that I am doing what I need to make my life as comfortable as possible. It's clear that he wants to make his life as comfortable as possible too, so perhaps I could listen to him and try to compromise a bit. However, humans need tools and animals to exploit to live. Humans, themselves do not have great strenght but we have knowladge and large brains. What we should conclude in the debate would be a good system of capitalism where the owner pays the workers a ``livable'' \textit{wage}.
        \item The old major says that the animals must never stand on two legs and act as humans.
    \end{enumerate}

\end{document}