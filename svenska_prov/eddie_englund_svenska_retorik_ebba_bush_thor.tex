\documentclass[12pt, letterpaper]{article}
\usepackage[utf8]{inputenc}
\usepackage{titlesec}

\title{Retorikanalys på Ebba Bush Thor}
\author{Eddie Englund}


\titleformat{\section}
{\large}
{}
{0em}
{}


\begin{document}
    \maketitle

    \section{}
    I ett tal som hölls under almedalsveckan av Ebba Bush Thor 2019. Ebba är Kristdemokraternas partiledare och har varit det sedan 2015 och innan det så satt hon I kommunalrådet i Uppsala. Almedalsveckan är ett politiskt evenemang varje och I Visby på Gotland. Dit kommer nästan alla partiledare och deras partier för att visa Sverige varför dom är viktiga och varför dom har rätt politik. Av precis samma anledning så var Ebba där och hade sitt argumenterande tal inför det Svenska folket.
    
    \section{}
    När man tittar på Ebbas tal så kan man snabbt se att hon har talat, skrivit många tidigare tal och att hon är mycket duktig på att både föra sig men också att leva in sig i det hon säger vilket är mycket bra för hennes etos. Dessutom så använder hon sig utav en rad olika stilfigurer för att fortsätta att förstärka sitt etos men också visa logos och patos. T.ex så tar hon upp olika problem i Sverige som sjukvårdsköer och hur folk dör i dom eftersom att dom är alldeles för långa och punkterar ut att det är den dåliga politiken driven av både dom röda partierna och gamla \textit{"allians vänner"} som har varit dålig för det svenska folket och deras bl.a sjukvård.

    Sjukvården specifikt är ett väldigt bra ämne att prata om eftersom att det är lätt att ta med logos med fakta om t.ex sjukvårdsköer och samtidigt få med patos i det eftersom att folk dör i det vilket genererar många känslor i folk som har upplevt systemets krackelerande väggar eller sett det hända nära och kära.

    \section{}
    Effekten av hennes tal och förstärkelsen av det kommer från hennes mycket goda användning utav stilfigurer som: anaforer (\textit{``Den statsminister som...''}),  metaforer (\textit{``Att låta politiken bottna i värderingar - i människosyn - det ger ett rättesnöre som gör att man kan skilja viktigt från mindre viktigt''}) och stegring

    \section{}

    Hennes talstruktur är också bra då hon börjar med att medela hennes budskap. Sedan så forsätter hon med att spekulera och skapa hopp och förtroende hos publiken och lystnarna. Hon forsätter att göra det genom att \textit{bevisa} att dom andra partierna (l, c, och allt vänster därifrån) är personerna som är skylldiga till det ``kaoset'' som vi har idag. 

    Hon avslutar sedan sitt tal med att fråga sina lystnare och sedan att förklara budskapet igen. Det budskap som hon har byggt upp under hela talet vilket är \textit{rösta oss vi har bäst värderingar och politik}.

\end{document}