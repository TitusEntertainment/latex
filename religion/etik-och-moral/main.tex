\documentclass[12pt]{article}

\title{Etik och Moral}
\author{Eddie Englund och Brian Nyberg}

\begin{document}
    

    \maketitle

    \section{Upphovsrätt}

    Upphovsrätten är iden att man ska kunna skydda sitt verk så att den personen kan bestämma vad som ska hända med det.

    \vspace{1cm}

    För:

    \begin{enumerate}
        \item Detta gör så att andra inte kan själa din ide och sälja det som sitt eget.
        \item Konsekvensen om det inte finns är att skaparen inte får kompensation för sitt arbete.
    \end{enumerate}

    \vspace{1cm}

    Mot:

    \begin{enumerate}    
        \item Om skaparen  har total kontroll över sitt verk så kan man till exmepel inte kritisera verket och visa det för att ge kontext.
        \item Vissa företag som till exmepel Hugo Boss har skickat ``Cease and desist'' brev till många företag som inkluderar order ``Boss'' i sitt namn. Alltså det kan bli dåligt för andra som har liknande saker. Detta kan man till exempel se i musik industrin då Ed Sheeran blev stämd för att han hade en liknande melodi som en annan artist i en av hanns låtar.
    \end{enumerate}
    

    \section{Stöld}

  
    För:

    \begin{enumerate}
        
        \item Sinnenslagsetik alltså anledning till att man gör det man gör tillåter det om det är kritiskt.
        \item Situationsetik    Handlingen anpassas till situationen  
    \end{enumerate}

    \vspace{1cm}

    Mot:

    \begin{enumerate}    
        
        \item pliktetik alltså det är olagligt
        \item Konsekvensetik alltså att någon kan bli skadad eller också att man förstör någon annas liv i processen av att förbättra sitt eget.
        \item Dygetik alltså för att bara en bra människa bör man inte göra det
    \end{enumerate}

   
\end{document}