\documentclass[12pt]{article}

\usepackage[utf8]{inputenc}
\usepackage[swedish]{babel}
\usepackage{relsize}
\usepackage{hyperref}
\usepackage[T1]{fontenc}
\usepackage{relsize}
\usepackage[backend=biber, style=alphabetic, sorting=ynt]{biblatex}
\usepackage{csquotes}
\usepackage{graphicx}
\usepackage{titlesec}
\usepackage{titling}
\usepackage{pgf-pie}
\usepackage{hyperref}


\bibliography{sources.bib}
\addbibresource{sources.bib}


\author{Eddie Englund \thanks{teinf17a}}
\title{Icke ekonomisk imigration}

\begin{document}
    
    \maketitle

    \section{Inledning}

    Under majoriten av Människans historia så har det funnits många krig och dåliga situationer. Men personerna i dom länderna tenderade att inte åka så långt, men dom som gjorde det åkte via båt till andra länder. Däremot så stannade majoriteten av dom I antingen samma krigsburna land eller så tar dom sig till andra när liggande länder.

    Men inte idag. Idag så kommer stora vågor av människor från områden som mellanöstern\cite{Guardian}. OwO

\end{document}
