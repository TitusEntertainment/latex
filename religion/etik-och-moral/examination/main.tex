\documentclass[12pt, onecolumn, letterpaper]{article}

\usepackage[utf8]{inputenc}
\usepackage[swedish]{babel}
\usepackage{relsize}
\usepackage{hyperref}
\usepackage[T1]{fontenc}
\usepackage{relsize}
\usepackage[backend=biber, style=numeric, sorting=ynt]{biblatex}
\usepackage{csquotes}
\usepackage{graphicx}
\usepackage{titlesec}
\usepackage{titling}
\usepackage{pgf-pie}
\usepackage{hyperref}


\bibliography{sources.bib}
\addbibresource{sources.bib}


\author{Eddie Englund Teinf17a}
\title{Icke ekonomisk imigration}

\begin{document}
    
    \maketitle

    \section{Inledning}

    Under majoriten av Människans historia så har det funnits många krig och dåliga situationer. Men personerna i dom länderna har inte haft den tendesen som vi ser idag. Dom fyr vädigt långt, men dom som gjorde det på den tiden åkte via båt till andra länder. Däremot så stannade majoriteten av dom I antingen samma krigsburna land eller så tar dom sig till andra när liggande länder.

    Men inte idag. Idag så kommer stora vågor av människor från områden som mellanöstern\cite{Guardian}. Men alla som kommer ifrån dessa länder kommer faktiskt inte från länder som är krigsburna men har ofta sämre ekonomier och länder i allmenhet. Dessa personer kommer eftersom att de vill ta del av det sociala systemet som vi har I Sverige.

    Men är det värkligen etiskt att ta in dessa människor?

    \section{Argumentation}


    I dagens Svenska samhället så är det vanligasta etiska argumentet kring immigration att det är humanitärt (dygetiskt) att ta in så många individer och deras familjer som vi kan. Vissa tycker detta eftersom att det kan vara så att dom som kommer har en dålig situation i deras land, alltså så utgår dessa personer utifrån en sinneslagsetisk ståndpunk. Det finns även dom som gör samma argument fast ur ett siuationsetiskt perspektiv i och med att det kan vara så att en eller flera familje medlämmare har blivit dödade av en diktatur eller liknande.

    Men, som med allting annat så finns det dom som tycker annorlunda. Till exempel så finns det en del partier i Sveriges regering som tycker att vi antigen ska skrota asylsökning helt, temporärt så att vi ska kunna \textit{fixa} våra interna problem i Sverige som t.ex sjukvårdsköerna. Detta argument kan anses som konsekvensetiskt i och med att man antar att dom myndigeheter, regerings uppgifter och infrastruktur inte kan hantera större våger av asylsökande eftersom att dom offta (inte alltid) inte har en bra utbildning eller inte kan språket ordentligt.

    I och med att det finnns många asyl sökande som inte är utbildade så kan det bli en större segregation i landet om dom inte blir integrerade ordentligt. Många säger att det redan har hänt idag tack vare den ökade brottsligheten speciellt mellan gäng\cite{Forskning&Framsteg}. På grund av detta så har bl.a bomb dåd blivit så många att ``Vi måste tyvär söka oss till krigszoner för att hitta motsvarighet''\cite{Expressen}. Dessa argument är ofta gjorda ur ett pliketiskt argument i och med att dom som för denna typ av politik är mycket logiska och inte emotionella(humanitärt före allt annat), så dom ser både en konsekvens men också en pliketiskt (bl.a ökade antal brot) anledning till att argumentera emot detta.


    \section{slutsatts}

    En högt debaterad fråga är vad som faktiskt är dygetisk gällande imigrations situationen. Är det mer dygetiskt att ta hand om det folk som redan finns i landet/nationen eller att ta in andra personer som behöver hjälp i andra länder? Ska man ha ett helt stängt eller ett helt öppna gränser? Det här är ganska svåra dillemmor. Personligen så tycker jag att den Svenska läran \textit{lagom} är det bästa sättet. Sverige har en historia av arbetsimmigration där det kommer personer som har liknande kultur och är utbildade som t.ex dom som kom under balkankrisen\cite{Wikipedia}. Den typen av immigration har fungerat väl eftersom att vi behöver arbetare eftersom att Svenskar inte föder tillräckligt med barn. Vi (enligt mig) bör (om vi då ska ta in personer) ta in personer som kommer från liknande kulturer t.ex inom EU's gränser. Dessutom så tycker jag att om vi på det bästa sättet ska hjälpa andra populationer så bör det vara i deras länder och inte här. Detta eftersom att ungefär 3\% av personen som åker/flyr från andra länder. Om vi skulle istället för att ta in dom här, hjälpa deras länder så skulle vi rädda \textbf{fler} personer en vad vi skulle om vi bara tog in dem.


    Så vad kan man få ut ifrån detta?
    Jo, man kan lära sig en ganska Svensk lära. \textit{Allting bör vara lagom}. Om man kan hitta ett mellan rum, alltså inte ta in alla men en del så kanshe man kan komma överens.
    Detta dillema visar också ett annat problem. Eftersom att vi är så olika så kommer vi också att ha olika etiska och moraliska åsikter vilket leder till låna debatter av denna typ.


    \printbibliography

\end{document}
